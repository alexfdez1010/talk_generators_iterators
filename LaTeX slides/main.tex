\documentclass[11pt]{beamer}
\usetheme{Madrid}
\usepackage[spanish]{babel}
\usepackage{minted}
\usepackage{amsmath}
\usepackage{amsfonts}
\usepackage{amssymb}
\usepackage{graphicx}
\usepackage{svg}
\usepackage{url}

\setbeamertemplate{navigation symbols}{}
\usecolortheme[named=orange]{structure}

\title{Generadores e iteradores en Python}
\author[Alejandro Fernández]{Alejandro Fernández Camello}
\institute{Python Coruña}
\logo{\includegraphics[height=0.8cm]{logo.jpg}}
\date{18 de noviembre de 2023} 

\AtBeginSection[]
{
  \begin{frame}
    \frametitle{Índice}
    \tableofcontents[currentsection]
  \end{frame}
}

\begin{document}

\begin{frame}
\titlepage
\end{frame}

\begin{frame}[fragile]

\begin{center} \Large
    ¿En qué se diferencian el código 1 y 2? 
\end{center} \normalsize

\begin{columns}[T]

\begin{column}{.45\textwidth}
\begin{block}{Código 1}
\begin{minted}[fontsize=\scriptsize]{python}
for i in list(range(1_000_000_000)):
    print(i)
\end{minted}
\end{block} 
\end{column}

\begin{column}{.45\textwidth}
\begin{block}{Código 2}
\begin{minted}[fontsize=\scriptsize]{python}
for i in range(1_000_000_000):
    print(i)
\end{minted}
\end{block}
\end{column}
  
\end{columns} 
\end{frame}

\begin{frame}{Índice}
\tableofcontents 
\end{frame}

\section{¿Por qué son útiles los iteradores y generadores?}
\begin{frame}{Ventajas de usar iteradores y generadores}
    \begin{itemize}
        \item <1-> Permiten la evaluación perezosa de datos (\textit{lazy evaluation}), generándolos solo cuando se requieren.
        \item <2-> Reducen la cantidad de memoria necesaria al almacenar menos datos.
        \item <3-> Facilitan el procesamiento de datos sin necesidad de generarlos todos de antemano.
        \item <4-> Simplifican el código y minimizan el riesgo de cometer errores.
    \end{itemize}
\end{frame}

\section{Iteradores}
\begin{frame}[fragile]{Definición}
\begin{block}{Implementación}
\begin{minted}[fontsize=\scriptsize]{python}
from collections.abc import Iterator

class MyIterator(Iterator):
    def __init__(self, *args, **kwargs):
        pass

    def __iter__(self):
        return self

    def __next__(self):
        pass
\end{minted}
\end{block}   
\end{frame}

\begin{frame}[fragile]{Ejemplos}
\begin{block}{Iterador sobre los cuadrados de los números naturales}
\begin{minted}[fontsize=\scriptsize]{python}
class SquareIterator(Iterator):
    def __init__(self):
        self.current = 1

    def __iter__(self):
        return self

    def __next__(self):
        self.current += 1
        return self.current ** 2

infinite_squares = SquareIterator()

for i, square in enumerate(infinite_squares, 1):
    print(f"{i} -> {square}")
\end{minted}
\end{block}   
\end{frame}

\begin{frame}[fragile]{\textit{Built-in} iteradores}
¿Qué iteradores trae ya definidos Python?
\begin{itemize}
    \item \texttt{range}
    \item \texttt{enumerate}
    \item \texttt{map}
    \item \texttt{filter}
\end{itemize}
\end{frame}

\begin{frame}[fragile]{\texttt{range}}
\begin{block}{Funcionamiento interno de \texttt{range}}
\begin{minted}[fontsize=\scriptsize]{python}
class RangeIterator:
    def __init__(self, start, stop, step=1):
        self.start, self.stop, self.step  = start, stop, step
        self.current = start - step

    def __iter__(self):
        return self

    def __next__(self):
        self.current += self.step
        if self.current >= self.stop:
            raise StopIteration
        return self.current

for i in range(0, 5, 2):
    print(i)
\end{minted}
\end{block}  
\end{frame}

\begin{frame}[fragile]{\texttt{enumerate}}
\begin{block}{Funcionamiento de \texttt{enumerate}}
\begin{minted}[fontsize=\scriptsize]{python}
class EnumerateIterator:
    def __init__(self, iterable):
        self.iterable = iter(iterable)
        self.index = -1

    def __iter__(self):
        return self

    def __next__(self):
        self.index += 1
        value = next(self.iterable)
        return self.index, value

for index, value in EnumerateIterator("abc"):
    print(index, value)
\end{minted}
\end{block}  
\end{frame}

\begin{frame}[fragile]{\texttt{map}}
\begin{block}{Funcionamiento de \texttt{map}}
\begin{minted}[fontsize=\scriptsize]{python}
class MapIterator:
    def __init__(self, function, iterable):
        self.function = function
        self.iterable = iter(iterable)

    def __iter__(self):
        return self

    def __next__(self):
        value = next(self.iterable)
        return self.function(value)

for value in MapIterator(lambda x: x * 2, [1, 2, 3]):
    print(value)
\end{minted}
\end{block}  
\end{frame}

\begin{frame}[fragile]{\texttt{filter}}
\begin{block}{Funcionamiento de \texttt{filter}}
\begin{minted}[fontsize=\scriptsize]{python}
class FilterIterator:
    def __init__(self, function, iterable):
        self.function = function
        self.iterable = iter(iterable)

    def __iter__(self):
        return self

    def __next__(self):
        while True:
            value = next(self.iterable)
            if self.function(value):
                return value

for value in FilterIterator(lambda x: x % 2 == 0, [1, 2, 3, 4]):
    print(value)
\end{minted}
\end{block}  
\end{frame}

\begin{frame}{Reflexión}
    \begin{center}
        ¿No estamos teniendo que escribir demasiado código?
    \end{center}
    \begin{center}
        ¿Por qué?
    \end{center}
\end{frame}

\begin{frame}{La gestión del estado}
    \begin{center}
        ¿No hay alguna manera de delegar la gestión del estado al propio programa?
    \end{center}
    \begin{center}
        Sí la hay, ¡los generadores!
    \end{center}
\end{frame}

\section{Generadores}

\begin{frame}[fragile]{Generadores al rescate}
\begin{columns}[T]

\begin{column}{.45\textwidth}
\begin{block}{Iterador}
\begin{minted}[fontsize=\scriptsize]{python}
class FibonacciIterator:
    def __init__(self, max_value):
        self.max_value = max_value
        self.a, self.b = 0, 1

    def __iter__(self):
        return self

    def __next__(self):
        self.a, self.b = self.b, \
        self.a + self.b
        if self.a > self.max_value:
            raise StopIteration
        return self.a
\end{minted}
\end{block} 
\end{column}

\begin{column}{.45\textwidth}
\begin{block}{Generador}
\begin{minted}[fontsize=\scriptsize]{python}
def fibonacci_generator(max_value):
    a, b = 0, 1
    while a <= max_value:
        yield a
        a, b = b, a + b
\end{minted}
\end{block}
\end{column}

\end{columns}
\end{frame}

\begin{frame}[fragile]{Usando generadores}
\begin{columns}[T]

\begin{column}{.45\textwidth}
\begin{block}{Iterador}
\begin{minted}[fontsize=\scriptsize]{python}
for fib in FibonacciIterator(21):
    print(i)
\end{minted}
\end{block} 
\end{column}

\begin{column}{.45\textwidth}
\begin{block}{Generador}
\begin{minted}[fontsize=\scriptsize]{python}
for fib in fibonacci_generator(21):
    print(i)
\end{minted}
\end{block}
\end{column}
  
\end{columns} 

\begin{center}
    ¡Se usan exactamente igual que los iteradores!
\end{center}
\end{frame}

\begin{frame}{La palabra mágica \texttt{yield}}
    \begin{itemize}
        \item Utilizar \texttt{yield} en lugar de \texttt{return} transforma automáticamente el código en un generador en lugar de una función.
        \item Al alcanzar \texttt{return} o \texttt{yield}, tanto funciones como generadores devuelven el control al punto del programa que los invocó.
        \item Si se invoca una función de nuevo, esta se reinicia desde el principio.
        \item En contraste, un generador reanuda la ejecución justo después del último \texttt{yield}.
    \end{itemize}
\end{frame}

\begin{frame}[fragile]{La palabra mágica \texttt{yield}}
\begin{block}{Generador}
\begin{minted}[fontsize=\scriptsize]{python}
def fibonacci_generator(max_value):
    a, b = 0, 1
    while a <= max_value:
        yield a
        a, b = b, a + b
\end{minted}
\end{block} 
\end{frame}

\begin{frame}[fragile]{Convertir listas a generadores}

\begin{center}
   Cuando usamos comprensión de listas hay una forma muy sencilla de convertirla en un generador. ¡Simplemente cambia los corchetes por paréntesis! 
\end{center}

\begin{block}{Listas a generadores}
\begin{minted}[fontsize=\scriptsize]{python}
squares_list = [x*x for x in range(1, 6)]
print(squares_list)

squares_gen = (x*x for x in range(1, 6))
print(squares_gen)
print(list(squares_gen))
\end{minted}

\begin{minted}[fontsize=\scriptsize]{text}
[1, 4, 9, 16, 25]
<generator object <genexpr> at 0x7fbcf1b8cc50>
[1, 4, 9, 16, 25]   
\end{minted}
\end{block} 
\end{frame}

\begin{frame}{¡Los generadores pueden hacer aún más cosas!}
    \begin{enumerate}
        \item Se les pueden mandar valores al generador por medio del método \texttt{send}.
        \item Si le añadimos un \texttt{return} podemos devolver valores al terminar la ejecución del generador.
    \end{enumerate}
\end{frame}

\begin{frame}[fragile]{Generador completo}
\begin{block}{Implementación}
\begin{minted}[fontsize=\tiny]{python}
def hello_n_times(n):
    for i in range(n):
        received = yield "Hello"
        if received: 
            print(f"Received: {received}")
    return f"I have printed {n} times hello"

gen = hello_n_times(3)
print(next(gen))
print(gen.send("World")) 
print(next(gen))  

try:
    next(gen)
except StopIteration as e:
    print(e.value)
\end{minted}
\begin{minted}[fontsize=\tiny]{text}
Hello
Received: World
Hello
Hello
I have printed 3 times hello 
\end{minted}
\end{block}
\end{frame}

\section{Conclusiones}

\begin{frame}{Conclusiones}
    \begin{itemize}
        \item Los iteradores y generadores contribuyen a una mayor eficiencia al reducir el uso de memoria.
        \item Permiten el cálculo de valores de manera secuencial en lugar de generarlos todos a la vez.
        \item Los generadores logran una funcionalidad similar a la de los iteradores, pero con código más conciso y menor complejidad.
        \item Los iteradores \textit{built-in} como \texttt{range}, \texttt{enumerate}, \texttt{map}, y \texttt{filter} son extremadamente útiles para simplificar el código.
    \end{itemize}
\end{frame}

\begin{frame}{Despedida}
    \begin{center}
        ¡Muchas gracias por haberme escuchado! 
    \end{center}
    \begin{center}
        Tenéis disponible la presentación y el código en el siguiente enlace: \url{https://github.com/alexfdez1010/talk_generators_iterators}
    \end{center}
    \begin{center}
        \includegraphics[width=0.33\textwidth]{qr.png}
    \end{center}
\end{frame}

\end{document}